\documentclass[12pt, letterpaper, oneside]{article}
\usepackage[margin=1in]{geometry}
\linespread{2.0}
\title{Snappy Title}
\author{Jamie Macdonald}
\begin{document}
\maketitle
Human life is bound to those forces which sustain it. We have a duty as actors of our planet to in turn sustain the machine that bears up our own particular ways of life. ``To Live With Herds'' (MacDougall 1968) is a cinematic survey of the broken machine that is the pastoral way of life in northeast Uganda. David MacDougall logically partitions the film into 5 to simply and tragically illustrate the imbalance struck by a conflict of ideals between a newly independent Uganda and the traditional tribal culture of the pastoral Jie people; ``Balance'', ``Changes'', ``The nation'', ``The value of cattle'', ``News from home''. This narrative is effectively characterised by our sense of place and culture in time, and represents a shared, broken model of the support structure required to sustain human life and culture.

We begin our experience with a view of the balance intended by the Jie pastoral way of life. We are first situated behind a mountain, Toror, and surrounded by neighbouring tribes existing beyond an unseeable distance through the recollection of a single native who gestures about the horizon. Then we are shown the daily chores of herd migrations and cultivation of food, told of sacrificial rites of passage, and the simple life desires of a teenager: find a girl to marry. Our first part is concluded with a quiet, lounging game of Mancala. We are then told of the elders' memories: a more peaceful time under British rule, the emerging of conflict following ikndependence. We see the nation's regulated food distribution, and the neglected, starving populations. We are told of the expectation for children to attend school, herders to pay taxes, and hence the necessity for money. We are brought to a cattle auction that puts a price on a hungry herder's cows, and shown the reliance of the Jie women on ``snuff''. This tragic spiral is punctuated with the travel of one Jie herder to the others in the mountains who ask him for news from home since their departure. He answers them: ``there is no news, just hunger at home''.

This tragic arc that David MacDougall deliberately draws is a testament to the film maker's power of persuasion. Indeed, very few words are needed to effectively indicate the key points of the woeful theme; we can fully realize the story by appealing to the compelling and concise lexicon of place, space, and culture. Yi-Fu Tuan distinguishes the first two terms in a strikingly simple interlude to his extensive treatise on Place and Space:
\begin{quote}
``space is formless and profane except for the sites that `stand out' because spirits are believed to dwell in them. These are the sacred places. They command awe. ... places, like human beings, acquire unique signatures in the course of time. ... Loosely speaking, the personality of place is a composite of natural endowment (the physique of the land) and the modifications wrought by successive generations of human beings. ... We can know a place subconsciously, through touch and remembered fragrances, unaided by the discriminating eye. While the eye takes in a lovely street scene and intelligence categorises it, our hand feels the iron of the school fence and stores subliminally its coolness and resistance in our memory. Through such modest hoards we can acquire in time a profound sense of place.''
\end{quote}
This definition on the one hand invokes a personal sense of awe and attachment to place in our own lives, while on the other hand casts space as its antonym: place devoid of meaning. It is clear that this discrepency is a highly subjective one; that is, a space is only a place to some class of people that shares a similar sense understanding and experience. Furthermore, as this sense directs our individual ways of life, we can consider such classes of individuals with shared understandings to be cultures.

Certainly, the Jie culture has a profound sense of place about and around their mountain, Toror. The Jie are charged with knowing where the more lush grazing lands are for their herds, and are thus in constant communication with their land. It is in particular the dry seasons that are hard on them, and every meteorological development has a direct impact on their livelihood. This is the point of the first part of five, to communicate the Jie sense of place to the viewer. It is effectively done by surveying a well distributed variety of settings; while the Jie herder situates us spatially from memory, we gain entry to share with him his sense of protection from the physique of the land: ``It is our mountain, Toror; it protects us.'', and a sketch of the land area where he grazes his herds; we share a sense of intimacy near the homestead watching the game of Mancala.

Hopefully without over-idealizing this fostered sense of place, we can give it the legitimacy it deserves, hence legitimizing the traditional way of life of the Jie. There is however a competing sense of place as seen through the eyes of the Ugandan nationalist; the nationalist foremost sees the importance in a sense of nationhood on the international level, undoubtedly more so than the Jie pastoralist. Lloyd Fallers on Ugandan in particular and post-British African nationalism in general, identifies that 
\begin{quote}
``the new nations, in order to achieve a degree of unity of purpose, need cultures which, first, will provide a measure of consensus among their diverse peoples and, second, will be capable of the constant innovation which existence in the modern world requires.''
\end{quote}
Fallers further identifies that this is ``in one of its aspects, ... a cultural problem''. The dilemma which I previously classified as a conflict of ideals between the Jie pastoralist and the Ugandan nationalist is, I assert, equivelantly characterized as the conflicting sense of place between the two parties. From the Jie's perspective, the nationalist sees their place as mere space, a neutral resource to be directed toward their own goals, and vice versa. Despite this characterization being a gross simplification of the situation, it is nonetheless efficient. The lens through which we see this conflict in ``To Live with Herds'' is the point-of-view of the Jie, and the tragedy is summarised by an oppressive force imposing regulation on the Jie people without regard for their balanced way of life, resulting in the downfall of Jie culture. This is yet another attestation to the film maker's power of persuasion. For instance, there is an undeniable sense of offense we share when we are told with disbelief by a Jie woman that they are required to now send their children to school - and to pay taxes to the state! 

If we are able to distance ourselves from this perspective, we can recognize that a characterization as a one-sided oppression is an oversimplification, that the nationalist's dream is as compelling to him as the pastoralist's is to her. What results is nonetheless doubly tragic: a nation divided in place. From this resulting perpective, we see that it is this mutual misunderstanding of place drives the narrative of the film, and represents the totality of resistances between the two parties; the enforcement of Ugandan schooling undermines the efficiency of the Jie herders, and the Jie's refusal to remain within prescribed boundaries undermines the nation's regulatory system, vital to a consistent economy of property.

If we are to fully understand the conflict of perception of place that causes the cultural divide in Uganda, we must regard the aerial view that details the motivations and strifes of both parties. While our exploration begins with the simple tragedy of the oppressed, a truly regained sense of balance for the Jie can only be achieved by allowing for shared experience and dialog of values with the nation. It is only through utmost respect for these values on both sides of the divide that the nation can know a unified sense of place:
\begin{quote}
We shall not cease from exploration\\
And the end of all our exploring\\
Will be to arrive where we started\\
And know the place for the first time.

T.S. Eliot, ``The Four Quartets''
\end{quote}

\end{document}